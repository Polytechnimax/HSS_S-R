\documentclass[a4paper, 12pt]{article}

\usepackage{fontspec}
\defaultfontfeatures{Mapping=tex-text}
\usepackage{xunicode}
\usepackage{xltxtra}

\usepackage[hidelinks]{hyperref}
\usepackage[francais]{babel}

%\usepackage[sectionmark, fancysections, twoside]{polytechnique}%titlepage

\title{Science et religion}
\author{Guillaume \textsc{Girol}, Maxime \textsc{Larcher}}
\begin{document}

\maketitle

\section*{Introduction}
L'Église catholique, depuis notamment le concile de Vatican II \cite{Vatican2} se préoccupe des aspects éthiques de la science et des utilisations techniques qui en découlent. Nous souhaitions savoir s'il en a toujours été ainsi.

Nous avons donc choisi de rechercher les réactions de l'Église à l'introduction de deux types d'armes directement liées à des avancées scientifiques : les gaz de combat pendant la 1\iere guerre mondiale d'une part, et la bombe atomique pendant la seconde guerre mondiale d'autre part.

Nous avons cherché des réactions d'une part dans des sources officielles, les encycliques et les textes des deux conciles de Vatican d'une part, et dans des sources non officielles mais susceptibles de traiter de sujets scientifiques et de leur répercussions éthiques, la \emph{Revue}\cite{Revue} des pères Jésuites.
\tableofcontents

\section{Les gaz de combat}
blah
\section{La bombe atomique}
chombier


%\cite{CryptoMili}

\bibliographystyle{alpha}
\bibliography{Science&religion}

\end{document}
